%%%%%%%%%%%%%%%%%%%%%%%%%%%%%%%%%%%%%%%%%%%%%%%%%%%%%%%%%%%%%%%%%%%%%%%
%%%-----------------------------------------------------------------%%%
%%%---------------------- IBARROND'S CORNER ------------------------%%%
%%%-----------------------------------------------------------------%%%
%%%                All the cool & portable LaTeX tricks             %%%
%%%%%%%%%%%%%%%%%%%%%%%%%%%%%%%%%%%%%%%%%%%%%%%%%%%%%%%%%%%%%%%%%%%%%%%
% Place right before \begin{document}

\pdfminorversion=7             % solve WARNING: PDF versions
\pdfsuppresswarningpagegroup=1 % solve WARNING: multiple pdfs page group

\usepackage{tikz}                  % display a tick with \checkmark
\def\checkmark{\tikz\fill[scale=0.4](0,.35) -- (.25,0) -- (1,.7) -- (.25,.15) -- cycle;} 

\usepackage{float}								 % figure/table position H
\usepackage{multirow}							 % merge rows in tables
\usepackage{url}									 % urls
\usepackage[utf8]{inputenc}				 % For accents in words and such
\usepackage[rightcaption]{sidecap}
\usepackage{booktabs} 						 % For prettier tables
\usepackage[margin= 1in]{geometry} % Document margins
\usepackage{fancyref}							 % Fancy references, \Fref & \fref
\usepackage{fancyhdr} % Headers and Footers (page numbers!)
\headheight=13.6pt
\pagestyle{fancy}     % Cool headers and foots
				% Remember to add \thispagestyle{empty}	on 1st page

\usepackage[nottoc, notlof, notlot]{tocbibind}% Include Bibliography in ref table


\newbox\tempbox										 % Nomenclature environment
\newenvironment{nomenclature}{%
   \newcommand\entry[2]{%
       \setbox\tempbox\hbox{##1.\quad}
       \hangindent\wd\tempbox\noindent{##1}\quad\ignorespaces##2\par}
       \subsubsection*{Nomenclature}}{\par\addvspace{12pt}}

% Notes, Missing references and TODOs																	
\newcommand{\note}[1]{\color{red}(#1!)\color{black}}
\newcommand{\missref}{\note{[REF]}}
\newcommand{\todo}[1]{\textcolor{blue}{[Todo: #1]}}

% Circled numbers, command \circled{1}
\usepackage{tikz}
\newcommand*\circled[1]{\tikz[baseline=(char.base)]{
            \node[shape=circle,draw,inner sep=2pt] (char) {#1};}}
						
						
% Fancy Captions
\usepackage{caption}
\captionsetup[figure]{labelfont={bf,it},textfont=it}
%\captionsetup[subfigure]{labelfont=bf,textfont=normalfont,singlelinecheck=off,justification=raggedright}
 
% New Math operator
\newcommand{\encr}[1]{\left \langle {#1} \right \rangle_{\mathbf{pub}}} 

% Paragraph spaces and indentation
% \setlength{\parindent}{0em}
%\setlength{\parskip}{2pt}

\usepackage{wrapfig}
% \begin{wrapfigure}[lineheight]{position}[overhang]{width}
% Positions
% r 	R 	right side of the text
% l 	L 	left side of the text
% i 	I 	inside edge near the binding (in a twoside document)
% o 	O 	outside edge far from the binding

% CODE SNIPPETS
\usepackage{listings}
\usepackage{color}

\definecolor{mygreen}{rgb}{0,0.6,0}
\definecolor{mygray}{rgb}{0.5,0.5,0.5}
\definecolor{mymauve}{rgb}{0.58,0,0.82}

\lstset{ 
  backgroundcolor=\color{white},   % choose the background color; you must add \usepackage{color} or \usepackage{xcolor}; should come as last argument
  basicstyle=\footnotesize,        % the size of the fonts that are used for the code
  breakatwhitespace=false,         % sets if automatic breaks should only happen at whitespace
  breaklines=true,                 % sets automatic line breaking
  captionpos=b,                    % sets the caption-position to bottom
  commentstyle=\color{mygreen},    % comment style
  escapeinside={\%*}{*)},          % if you want to add LaTeX within your code
  extendedchars=true,              % lets you use non-ASCII characters; for 8-bits encodings only, does not work with UTF-8
  frame=single,	                   % adds a frame around the code
  keepspaces=true,                 % keeps spaces in text, useful for keeping indentation of code (possibly needs columns=flexible)
  keywordstyle=\color{blue},       % keyword style
  language=Python,                 % the language of the code
  numbers=left,                    % where to put the line-numbers; possible values are (none, left, right)
  numbersep=5pt,                   % how far the line-numbers are from the code
  numberstyle=\tiny\color{mygray}, % the style that is used for the line-numbers
  rulecolor=\color{black},         % if not set, the frame-color may be changed on line-breaks within not-black text (e.g. comments (green here))
  showspaces=false,                % show spaces everywhere adding particular underscores; it overrides 'showstringspaces'
  showstringspaces=false,          % underline spaces within strings only
  showtabs=false,                  % show tabs within strings adding particular underscores
  stepnumber=2,                    % the step between two line-numbers. If it's 1, each line will be numbered
  stringstyle=\color{mymauve},     % string literal style
  tabsize=2,	                   % sets default tabsize to 2 spaces
  title=\lstname                   % show the filename of files included with \lstinputlisting; also try caption instead of title
}

\lstdefinestyle{customc}{
  belowcaptionskip=1\baselineskip,
  breaklines=true,
  frame=L,
  xleftmargin=\parindent,
  language=Python,
  showstringspaces=false,
  basicstyle=\footnotesize\ttfamily,
  keywordstyle=\bfseries\color{green!40!black},
  commentstyle=\itshape\color{purple!40!black},
  identifierstyle=\color{blue},
  stringstyle=\color{orange},
}

%%%%%%%%%%%%%%%%%%%%%%%%%%%%%%%%%%%%%%%%%%%%%%%%%%%%%%%%%%%%%%%%%%%%%%%
%%%------------------- END OF IBARROND'S CORNER --------------------%%%
%%%%%%%%%%%%%%%%%%%%%%%%%%%%%%%%%%%%%%%%%%%%%%%%%%%%%%%%%%%%%%%%%%%%%%%
